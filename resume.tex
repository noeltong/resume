% !TEX program = xelatex

\documentclass{resume}
%\usepackage{zh_CN-Adobefonts_external} % Simplified Chinese Support using external fonts (./fonts/zh_CN-Adobe/)
%\usepackage{zh_CN-Adobefonts_internal} % Simplified Chinese Support using system fonts

\usepackage{iitem}
\usepackage{hyperref}

\begin{document}
\pagenumbering{gobble} % suppress displaying page number

\name{Shangqing Tong}

\basicInfo{
  \email{tongshq@shanghaitech.edu.cn} \textperiodcentered\ 
  \phone{(+86) 188-0031-8979} \textperiodcentered\ 
  \github[\texttt{github.com/noeltong}]{https://github.com/noeltong}}

\section{Education}
% \section{\faGraduationCap\quad Education}
\datedsubsection{\textbf{ShanghaiTech University}, Pudong, Shanghai, China}{2021 -- Present}
\textit{Master student} in Electronics Engineering (EE), expected March 2024
\datedsubsection{\textbf{Jiangnan University}, Wuxi, Jiangsu, China}{2017 -- 2021}
\textit{B.S.} in Communication Engineering

\section{Publications}
% \section{\faUsers\quad Publications}
\datedsubsection{\textbf{Score-based Generative Models for Photoacoustic Image Reconstruction with Rotation Consistency Constraints}}{}
\textit{IEEE Transactions on Medical Imaging, IF 10.6, Under Review}
\role{First Author}{Colaborated with Dr.~Hengrong Lan, Department of Biomedical Engineering, School of Medicine, Tsinghua University.}
Reconstructing photoacoustic tomography images with score-based diffusion models.
\begin{itemize}
  \item Trained with only ground truth images, our method achieved competitive performance against supervised U-Net, while ours has higher generalization capability;
  \item A constraint is designed following the rotation equivariance between PAT measurements and images, which is used to guide the Langevin sampling process;
  \item Our method achieved 35.06 PSNR, 0.913 SSIM in uniform sampling with 32 measurements; and 29.69 PSNR, 0.823 SSIM in limited view with 32 measurements (128 in total).
\end{itemize}

\datedsubsection{\textbf{Review of photoacoustic imaging plus X}}{}
\textit{Journal of Biomedical Optics, IF 3.5, Under Review}
\role{Co-first Author}{Work with teammates in Hybrid Imaging System Laboratory, ShanghaiTech University.}
A brief review of deep learning applications in photoacoustic imaging among recent years.

\datedsubsection{\textbf{Interpreting Infrared Thermography with Deep Learning to Assess the Mortality Risk of Critically Ill Patients at Risk of Hypoperfusion}}{}
\textit{Reviews in Cardiovascular Medicine, IF 2.7}
\role{Co-first Author}{Colaborated with Department of Critical Care Medicine, Zhongshan Hospital, Fudan University.}
The aim of this study was to assess the mortality risk of critically ill patients at risk of hypoperfusion in a prospective cohort by infrared thermography combined with deep learning methods.
\begin{itemize}
  \item Compared the classification capability of several widely used vision backbones;
  \item Combined conventional cross-entropy loss with focal loss and label smoothing, which further improved the performance on the imbanlanced dataset.
\end{itemize}

\datedsubsection{\textbf{Benign and Malignant Classification of Human Colorectal Tissue by Acoustic-Resolution Photoacoustic Microscopy}}{}
\textit{2022 IEEE International Ultrasonics Symposium, Oral Presentation}
\role{Co-first Author}{Colaborated with the First Medical Centre, Chinese PLA General Hospital.}
Classifying the benign and malignant tissues using wavelet transform of the PAM signals.
\begin{itemize}
  \item Signals of cancer, polyp and normal tissues were obtained by scanning the \textit{ex-vivo} colorectal samples;
  \item Classify the cancer and normal regions with wavelet transform.
\end{itemize}

\datedsubsection{\textbf{Machine-Learning-based Colorectal Tissue Classification via Acoustic Resolution Photoacoustic Microscopy}}{}
\textit{Preprint, arXiv:2307.08556}
\role{First Author}{Colaborated with the First Medical Centre, Chinese PLA General Hospital.}
Comparison of the classification capability of several machine learning algorithms using the PAM signals of benign and malignant colorectal tissues.
\begin{itemize}
  \item Further improved the signal dataset, involving more tissues and improved the accuracy of the dataset;
  \item Classify the cancer and normal signals with several machine learning algorithms, with several metrics calculated and compared.
\end{itemize}

\datedsubsection{\textbf{Frequency-selected Adaptive Matched Filter De-noising for Photoacoustic Imaging}}{}
\textit{2023 IEEE International Ultrasonics Symposium, Oral Presentation, In Press}
\role{Third Author}{Work with teammates in Hybrid Imaging System Laboratory, ShanghaiTech University.}
PAM signal denoising with designed matched filter.

% Reference Test
%\datedsubsection{\textbf{Paper Title\cite{zaharia2012resilient}}}{May. 2015}
%An xxx optimized for xxx\cite{verma2015large}
%\begin{itemize}
%  \item main contribution
%\end{itemize}

\section{Skills}
% \section{\faCogs\quad Skills}
\begin{itemize}
    \item Research Interests
    \begin{itemize}
        \item Solving inverse problems in medical imaging with generative models,
        \item Self-supervised representation learning and other pre-training methods,
        \item Computer vision in biomedical applications.
    \end{itemize}
    \item Platform: Linux.
    \item Programming Languages: Python, MATLAB, \LaTeX{}.
    \item DL framework: PyTorch > JAX.
\end{itemize}

\section{Internship}
% \section{\faHeartO\quad Internship}
\datedline{\textbf{Central Research Institute, United Imaging Healthcare Group}, Shanghai}{Oct. 2023 -- Present}
During my internship at United Imaging, my work mainly focused on the reconstruction and enhancement of positron emission tomography images using generative models like diffusion-based models.


\section{Miscellaneous}
% \section{\faInfo\quad\ Miscellaneous}
\begin{itemize}
  \item \href{https://scholar.google.com/citations?user=-TaP8h4AAAAJ&hl=zh-CN}{Link to my Google Scholar profile}.
  \item Languages
  \begin{itemize}
    \item English: Fluent (CET-4 610, CET-6 509),
    \item Mandarin: Native speaker.
  \end{itemize}
\end{itemize}

%% Reference
%\newpage
%\bibliographystyle{IEEETran}
%\bibliography{mycite}
\end{document}
