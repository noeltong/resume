% !TEX program = xelatex

\documentclass{resume}
%\usepackage{zh_CN-Adobefonts_external} % Simplified Chinese Support using external fonts (./fonts/zh_CN-Adobe/)
%\usepackage{zh_CN-Adobefonts_internal} % Simplified Chinese Support using system fonts

\usepackage{iitem}
\usepackage{hyperref}

\begin{document}
\pagenumbering{gobble} % suppress displaying page number

\name{Shangqing Tong}

\basicInfo{
  \email{tongshq@shanghaitech.edu.cn} \textperiodcentered\ 
  \phone{(+86) 188-0031-8979} \textperiodcentered\ 
  \github[\texttt{github.com/noeltong}]{https://github.com/noeltong}}

\section{Education}
% \section{\faGraduationCap\quad Education}
\datedsubsection{\textbf{ShanghaiTech University}, Pudong, Shanghai, China}{2021 -- Present}
\textit{Master student} in Electronics Engineering (EE), expected March 2024
\datedsubsection{\textbf{Jiangnan University}, Wuxi, Jiangsu, China}{2017 -- 2021}
\textit{B.S.} in Communication Engineering

\section{Publications}
% \section{\faUsers\quad Publications}
\datedsubsection{\textbf{Enhancing Sparse Photoacoustic Tomography Reconstruction with Score-based Generative Models}}{}
\textit{2024 IEEE International Symposium on Biomedical Imaging, In Press}
\role{First Author}{Collaborated with Dr.~Hengrong Lan, School of Biomedical Engineering, Tsinghua University.}
Reconstructing sparse photoacoustic tomography images from limited projections with uniform downsampling using score-based diffusion models.
\begin{itemize}
  \item Trained with only ground truth images, our method achieved competitive performance against supervised U-Net, while ours could generalize to different numbers of projections;
  \item A constraint is designed following the rotation equivariance between photoacoustic tomography measurements and images, which is used to guide the vanilla diffusion sampler.
%   \item Our method achieved 35.06 PSNR, 0.913 SSIM in uniform sampling with 32 measurements; and 29.69 PSNR (128 in total).
\end{itemize}

\datedsubsection{\textbf{Exploring Geometric Characteristics for Enhanced Photoacoustic Tomography Reconstruction with Diffusion Models}}{}
\textit{Medical Image Analysis, IF 10.9, Under Review}
\role{First Author}{Collaborated with Dr.~Hengrong Lan, School of Biomedical Engineering, Tsinghua University.}
A extended journal version of our paper published in \textit{2024 IEEE International Symposium on Biomedical Imaging}, with more detailed explanation of our method and more experiments. We proved that our method could recover limited view downsampling images.

\datedsubsection{\textbf{Photoacoustic Imaging Plus X: A Review}}{}
\textit{Journal of Biomedical Optics, IF 3.5}
\role{First Author}{Work with teammates in Hybrid Imaging System Laboratory, ShanghaiTech University.}
A brief review of deep learning applications in photoacoustic imaging among recent years (Section 6 of the review paper).

\datedsubsection{\textbf{Interpreting Infrared Thermography with Deep Learning to Assess the Mortality Risk of Critically Ill Patients at Risk of Hypoperfusion}}{}
\textit{Reviews in Cardiovascular Medicine, IF 2.7}
\role{First Author}{Collaborated with Department of Critical Care Medicine, Zhongshan Hospital, Fudan University.}
The aim of this study was to assess the mortality risk of critically ill patients at risk of hypoperfusion in a prospective cohort by infrared thermography combined with deep learning methods.
\begin{itemize}
  \item Compared the classification capability of several widely used vision backbones;
  \item Combined conventional cross-entropy loss with focal loss and label smoothing, which further improved the performance on the imbalanced dataset.
\end{itemize}

\datedsubsection{\textbf{Benign and Malignant Classification of Human Colorectal Tissue by Acoustic-Resolution Photoacoustic Microscopy}}{}
\textit{2022 IEEE International Ultrasonics Symposium, Oral Presentation}
\role{First Author}{Collaborated with the First Medical Centre, Chinese PLA General Hospital.}
Classifying the benign and malignant tissues using wavelet transform of the photoacoustic microscopy signals.
\begin{itemize}
  \item Signals of cancer, polyp and normal tissues were obtained by scanning the \textit{ex-vivo} colorectal samples;
  \item Classify the cancer and normal regions with wavelet transform.
\end{itemize}

\datedsubsection{\textbf{Frequency-selected Adaptive Matched Filter De-noising for Photoacoustic Imaging}}{}
\textit{2023 IEEE International Ultrasonics Symposium, Oral Presentation}
\role{Third Author}{Work with teammates in Hybrid Imaging System Laboratory, ShanghaiTech University.}
Photoacoustic microscopy signal denoising with a designed matched filter.

\datedsubsection{\textbf{TFDT-Net: Improving Photoacoustic Imaging Reconstruction with Learnable Filter Transformer}}{}
\textit{2024 IEEE International Symposium on Biomedical Imaging, In Press}
\role{Third Author}{Work with teammates in Hybrid Imaging System Laboratory, ShanghaiTech University.}
Reconstructing high quality photoacoustic tomography images with transformers.

\datedsubsection{\textbf{Self-Supervised Neural Network for Patlak-Based Parametric Imaging in Clinical Dynamic Total-body $^{18}$F-FDG PET}}{}
\textit{IEEE Transactions on Medical Imaging, IF 10.6, Under Review}
\role{Fourth Author}{Work done during an internship at Central Research Institute, United Imaging Healthcare.}
Using partial dynamic positron emission tomography volumes to predict high quality Patlak parametric images of patient data with $^{18}$F-FDG injection.

\datedsubsection{\textbf{Input function separation in dual-tracer $^{18}$F-FDG and $^{68}$Ga-FAPI-04 single dynamic PET session using a recurrent neural network}}{}
\textit{2024 The Society of Nuclear Medicine and Molecular Imaging Annual Meeting, Under Review}
\role{Third Author}{Work done during an internship at Central Research Institute, United Imaging Healthcare.}
Input function of $^{18}$F-FDG and $^{68}$Ga-FAPI-04 separation with a Long Short-Term Memory (LSTM) network

\datedsubsection{\textbf{A Self-Supervised Neural Network with Patlak analysis for $^{18}$F-FDG Total-body PET Parametric Imaging}}{}
\textit{2024 The Society of Nuclear Medicine and Molecular Imaging Annual Meeting, Under Review}
\role{Third Author}{Work done during an internship at Central Research Institute, United Imaging Healthcare.}
Self-supervised prediction of Patlak parametric images using partial volumes of dynamic positron emission tomography scans.

\datedsubsection{\textbf{Machine-Learning-based Colorectal Tissue Classification via Acoustic Resolution Photoacoustic Microscopy}}{}
\textit{Preprint, arXiv:2307.08556}
\role{First Author}{Collaborated with the First Medical Centre, Chinese PLA General Hospital.}
Comparison of the classification capability of several machine learning algorithms using the photoacoustic microscopy signals of benign and malignant colorectal tissues. This work is a further study of our companion work published in \textit{2022 IEEE International Ultrasonics Symposium}.
\begin{itemize}
  \item The signal dataset was further improved, involving more tissue samples and accuracy;
  \item Classification of the cancer and normal signals using several machine learning algorithms, with several metrics calculated and compared.
\end{itemize}

% Reference Test
%\datedsubsection{\textbf{Paper Title\cite{zaharia2012resilient}}}{May. 2015}
%An xxx optimized for xxx\cite{verma2015large}
%\begin{itemize}
%  \item main contribution
%\end{itemize}

\section{Skills}
% \section{\faCogs\quad Skills}
\begin{itemize}
    \item Research Interests
    \begin{itemize}
        \item Solving inverse problems in medical imaging (particularly in photoacoustic imaging) with generative models,
        \item Self-supervised representation learning and other pre-training methods,
        \item Computer vision in biomedical applications.
    \end{itemize}
    \item Platform: Linux > Windows.
    \item Programming Languages: Python, MATLAB, \LaTeX{}.
    \item DL framework: PyTorch > JAX.
\end{itemize}

\section{Internship}
% \section{\faHeartO\quad Internship}
\datedline{\textbf{Central Research Institute, United Imaging Healthcare Co., LTD}, Shanghai}{Oct. 2023 -- Present}
During my internship at Central Research Institute of United Imaging, my work mainly focused on the quantitative analysis algorithms of parametric imaging of dynamic positron emission tomography (dPET) imaging with patient data from uEXPLORER, a total body PET/CT scanner, under the supervision of Prof.~Yun Zhou. My work mainly contains
\begin{itemize}
    \item Dual-tracer time-activity curve separation with machine learning algorithms;
    \item Accelerating irreversible 2-tissue compartment model with CUDA and PyTorch;
    \item Development of automatic PET/CT parametric imaging and time-activity curve calculation (including regions of interest of brain and tissues of total body) pipeline with CT segmentation.
\end{itemize}


\section{Miscellaneous}
% \section{\faInfo\quad\ Miscellaneous}
\begin{itemize}
  \item \href{https://scholar.google.com/citations?user=-TaP8h4AAAAJ&hl=zh-CN}{My Google Scholar profile}.
  \item Languages
  \begin{itemize}
    \item English: Fluent (CET-4 610, CET-6 504),
    \item Mandarin: Native speaker.
  \end{itemize}
  \item Prof.~Fei Gao (\url{gaofei@shanghaitech.edu.cn}), Prof.~Yun Zhou (\url{yun.zhou@united-imaging.com}).
\end{itemize}

%% Reference
%\newpage
%\bibliographystyle{IEEETran}
%\bibliography{mycite}
\end{document}
