% !TEX TS-program = xelatex
% !TEX encoding = UTF-8 Unicode
% !Mode:: "TeX:UTF-8"

\documentclass{resume}
\usepackage{zh_CN-Adobefonts_external} % Simplified Chinese Support using external fonts (./fonts/zh_CN-Adobe/)
% \usepackage{NotoSansSC_external}
% \usepackage{NotoSerifCJKsc_external}
% \usepackage{zh_CN-Adobefonts_internal} % Simplified Chinese Support using system fonts
\usepackage{linespacing_fix} % disable extra space before next section
\usepackage{cite}

\begin{document}
\pagenumbering{gobble} % suppress displaying page number

\name{童尚清}

\basicInfo{
  \email{tongshq@shanghaitech.edu.cn} \textperiodcentered\ 
  \phone{(+86) 188-0031-8979} \textperiodcentered\ 
  \github[\texttt{github.com/noeltong}]{https://github.com/noeltong}}
 
\section{\faGraduationCap\quad 教育背景}
\datedsubsection{\textbf{上海科技大学}, 上海, 浦东新区}{2021 -- 至今}
\textit{在读硕士研究生}\quad 电子科学与技术, 预计2024年6月毕业
\datedsubsection{\textbf{江南大学}, 江苏, 无锡}{2017 -- 2021}
\textit{学士}\quad 通信工程

\section{\faUsers\quad 科研经历}
\datedsubsection{\textbf{基于扩散模型的PAT重建算法}}{IEEE TMI (2023, 在投)}
\role{共同一作}{与清华大学兰恒荣博士合作}
使用扩散模型实现无监督光声层析成像重建任务
\begin{itemize}
    \item 只使用全采样图片进行训练, 实现了与监督训练的U-Net相近的性能并有更高的泛化能力;
    \item 根据PAT图片与信号的旋转不变性设计了约束项, 引导郎之万动力学采样过程;
    \item 均匀降采样, 随机降采样条件下结果为(...). 有限视角下的结果明显优于监督训练的基线模型.
\end{itemize}

\datedsubsection{\textbf{ICU重症病人死亡率预测}}{Reviews in Cardiovascular Medicine (2023)}
\role{共同一作}{与复旦大学附属中山医院重症医学科合作}

\begin{itemize}
    \item 根据ICU重症病人的腿部红外热图数据预测重症病人的死亡率, 比较了多种通用视觉主干的性能优劣;
    \item 结合了Focal Loss, 在不平衡数据集上得到更高的准确率.
\end{itemize}

\datedsubsection{\textbf{PAM结直肠癌信号处理与诊断}}{IEEE IUS (2022), 口头报告}
\role{共同一作}{与中国人民解放军总医院第一医学中心普通外科医学部合作}

\begin{itemize}
    \item 使用PAM对离体结直肠癌组织样品进行扫描, 得到了癌症, 息肉与健康组织的PAM信号;
    \item 使用信号的频谱分析方法将癌症组织信号与健康组织信号加以区分;
    \item 比较了多种机器学习方法对癌症与健康信号的分类性能.
\end{itemize}

\section{\faCogs\quad 技能}
% increase linespacing [parsep=0.5ex]
\begin{itemize}[parsep=0.5ex]
  \item 编程语言: 熟练掌握Python, MATLAB, \LaTeX\
  \item DL工具: PyTorch, JAX
  \item 平台: Linux
\end{itemize}

\section{\faHeartO\quad 获奖情况}
\datedline{上海科技大学研究生校级奖学金}{2021 -- 2024}
% \datedline{其他奖项}{2015}

\section{\faInfo\quad 其他}
% increase linespacing [parsep=0.5ex]
\begin{itemize}[parsep=0.5ex]
  \item 谷歌学术: \url{https://scholar.google.com/citations?user=-TaP8h4AAAAJ&hl=zh-CN}
  \item 语言: 英语 - 熟练(四级610, 六级509)
\end{itemize}

%% Reference
%\newpage
%\bibliographystyle{IEEETran}
%\bibliography{mycite}
\end{document}
